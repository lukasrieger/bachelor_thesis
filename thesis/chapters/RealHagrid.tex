\section{Testing the real-world implementation of Hagrid}
\label{sec:real_hagrid}
Given that we have successfully developed a framework in which we can test our abstract model of Hagrid, we decided to integrate the official implementation of Hagrid into our testing efforts.
Assuming that the maintainers of Hagrid would likely not appreciate it if their server received hundreds of spam-like calls to their server at \url{https://keys.openpgp.org/}, running our tests requires a separate instance of Hagrid to be running on the local machine.

\paragraph{Communicating with Hagrid}
The main effort of integrating Hagrid into our implementation focussed on the relaying and receiving of messages between our Scala frontend and Hagrid.
To this end, we use \emph{Sttp}\footnote{https://sttp.softwaremill.com/en/latest/}, which is a simple library that enables us to communicate with the server through REST-full HTTP calls. The transferred data is being encoded as \emph{JSON}, for which we rely on \emph{Circe}\footnote{https://circe.github.io/circe/} as a dependency.

HAGRID itself defines a specific interface, through which external applications may communicate with the PGP server. This interface, called \emph{Verifying Keyserver (VKS) Interface}\footnote{https://keys.openpgp.org/about/api}, exposes the following endpoints: 
\begin{figure}[h]
    \begin{itemize}
        \item \mintinline{text}{GET}  \mintinline{text}{/vks/v1/by-fingerprint/FINGERPRINT}
        \item \mintinline{text}{GET}  \mintinline{text}{/vks/v1/by-keyid/KEY-ID}
        \item \mintinline{text}{GET}  \mintinline{text}{/vks/v1/by-email/URI-ENCODED EMAIL-ADDRESS}
        \item \mintinline{text}{POST} \mintinline{text}{/vks/v1/upload}
        \item \mintinline{text}{POST} \mintinline{text}{/vks/v1/request-verify}
    \end{itemize}
    \caption{HTTP endpoints defined by Hagrid}
\end{figure}

In contrast to our abstract server model outlined in chapter \ref{sec:abstract_model}, this implementation does not maintain any local state on its own. Instead, we define a new implementation of \mintinline{scala}{HagridInterface} which simply relays all method calls to Hagrid via HTTP.

As an example, lets look at the \mintinline{scala}{upload} method: 
\begin{code}
    \begin{minted}{scala}
    override def upload(key: Key): Token = {
        val response = basicRequest
            .post(hag("/vks/v1/upload"))
            .body(UploadBody(key.armored))
            .response(asJson[UploadResponse])
            .send()
            .body
    
        Token(response
        .toOption
        .get
        .token)
        }
    \end{minted}
    \caption{Uploading a PGP key to the real Hagrid server}
\end{code}
\label{code:hagrid_upload}
First, we construct a \emph{POST}-Request which targets the \emph{VKS} endpoint responsible for uploading PGP keys. We then attach the \emph{armored} key-text as the request body and finally instruct the request to encode any eventual response as an instance of \mintinline{scala}{UploadResponse}.
The most crucial bit of information included in the \mintinline{scala}{UploadResponse} is the token, which we require in order to send any verification requests at a later point.

Note that our implementation at this moment does not have any robust error handling mechanism and this generally \emph{ignores} the possibility of an error occurring while communicating with Hagrid. We are aware that his my cause unforeseen problems in some edge-cases, like the local Hagrid instance suddenly crashing.

\subsection{Receiving verification mails from HAGRID}
While we were able to simulate the transmission of verification mails in an abstract way for our server model, we had no such option when dealing with the real-world Hagrid server. 

During the  earlier development phases, we assumed that we would be forced to provide a dummy implementation of \mintinline{bash}{sendmail}, as the documentation claimed that HAGRID would require a working \mintinline{bash}{sendmail} configuration to be available in the environment in order to send and receive emails.
Luckily, we discovered that HAGRID also provides an alternative mode of operation, in which all email transfer is handled directly through the filesystem.
Hagrid uses a configuration file called \mintinline{text}{Rocket.toml}, which allows us to specify a folder to be used as the destination for all outgoing Hagrid emails.   
Modifying the configuration file by adding the line \mintinline{text}{mail_folder=/PATH/TO/DESTINATION} therefore causes HAGRID to write all emails to a unique file in the specified folder.

This allows us to simply observe the given directory and react to any file modifications within the folder.
More specifically, we define a method, that blocks its execution until it has successfully read the expected amount of emails: 
\begin{code}
    \begin{minted}{scala}
    def consumeMail(expectedSize: Int): Seq[Body] = {
        val watchKey = mailWatcher.take
        val events = watchKey.pollEvents.asScala
        val updatePaths = events
            .take(expectedSize)
            .map(_.asInstanceOf[WatchEvent[Path]])
            .map(_.context())
    
        val mails = updatePaths.flatMap { currentPath =>
            val resolved = mailPath.resolve(currentPath).toFile
            val source = Source.fromFile(resolved)
            val bodies: Seq[Body] = parseMail(source.mkString)
            source.close
            resolved.delete
            val isValid = watchKey.reset
            bodies
        }
        mails
    }
    \end{minted}
    \caption{Reading Hagrid's emails by observing filesystem changes}
\end{code}
\label{code:consumeMail}
where the returned sequence of mail bodies takes the form of:
\begin{minted}{scala}
    case class Body(
        fingerprint: Fingerprint, 
        token: Token, 
        identity: Identity
    )
\end{minted}

Based on this helper method, we can implement \mintinline{scala}{requestVerify}, which sends a verification request and waits for a corresponding verification email:
\begin{code}
    \begin{minted}{scala}
    def requestVerify(from: Token, emails: Set[PgpIdentity]) = {
        if (emails.isEmpty) return Seq.empty
        val response = basicRequest
            .post(hag("/vks/v1/request-verify"))
            .body(VerifyRequest(from.uuid, 
                  emails.map(_.email).toList)
                 ).send()
            .body.toOption

        val shouldReadMail = checkResponse(
            emails.map(_.email), 
            response)

        (response, shouldReadMail) match {
            case (Some(x), true) => consumeMail(emails.size)
            case _ => Seq.empty
        }
    }
    \end{minted}
    \caption{Requesting verification of a set of identities}
\end{code}

As one can see, we make use of a helper method \mintinline{scala}{checkResponse} which parses the server response and checks whether our verification request has had any effect. This is important because Hagrid will ignore requests for verification of already verified identities and thus skip the process of sending an additional verification email. If we simply ignored this, \mintinline{scala}{consumeMail} would wait indefinitely for a corresponding email that will never be sent. 

\paragraph{Parsing plain text confirmation mails}
To consume the emails within the selected directory and use the information contained within them, we had to parse the plain text content of said emails and extract the relevant information. In both cases of identity confirmation or revocation, these relevant bits of data would be the \mintinline{scala}{Fingerprint}, the \mintinline{scala}{Token} and the respective \mintinline{scala}{Identity} for which the email was issued. For example, a typical email issued by Hagrid in response to a confirmation request might look like this: 
\begin{figure}[H]
    \begin{minted}{text}
    To: <ISSUE_EMAIL@ADDRESS> 
    Hi,
    
    Dies ist eine automatische Nachricht von localhost.
    Falls dies unerwartet ist, bitte die Nachricht ignorieren.

    OpenPGP Schlüssel: 23B2E0C54487F50AC59134C3A1EC9765D7B25C5A 
    Damit der Schlüssel über die Email-Adresse "<ISSUE_EMAIL@ADDRESS>" 
    gefunden werden kann, klicke den folgenden Link:
    
    http://localhost:8080/verify/vrTohvV8q552KMvARBE7foqkvGtUrfDl3iiyX9yqeOX 
    
    Weitere Informationen findest du unter http://localhost:8080/about
    \end{minted}
    \caption{Automatically generated verification email by Hagrid}
\end{figure}

We use a regular expression to parse the relevant information from these mails. This functionality is summarized in a method \mintinline{scala}{parseMail}:
\begin{code}
    \begin{minted}{scala}
    val VERIFY_PATTERN: Regex = 
        "(?s).*To: <(\\S+)>.*OpenPGP key: ||
        (\\S+).*http://localhost:8080/verify/(\\S+).*<!doctype html>.*".r

    def parseMail(mail: String): Seq[Body] =
        decode[HagridMail](mail)
            .map(mail => new String(mail.message))
            .map {
            case REVOKE_PATTERN(identity, fingerprint, token) => 
                Body(FingerprintImpl(fingerprint), 
                        Token(token), 
                        PgpIdentity(identity)
                    )
            case VERIFY_PATTERN(identity, fingerprint, token) => 
                Body(FingerprintImpl(fingerprint), 
                        Token(token), 
                        PgpIdentity(identity)
                    )
            }
            .toSeq
    \end{minted}
    \caption{Parsing relevant information from Hagrid emails}
\end{code}

As seen in the code sample in chapter \ref{code:consumeMail}, this function is responsible for parsing the mails read by \mintinline{scala}{consumeMail} from the dedicated mail directory.

\subsection{Generating valid PGP keys}
In the context of our abstract Hagrid model we were able to rely on a simplified implementation of our PGP \mintinline{scala}{Key} type: 
\begin{minted}{scala}
sealed trait Key {
    def armored: String
    def keyId: KeyId
    def fingerprint: Fingerprint
    def identities: Set[Identity]
    def restrictedTo(ids: Set[Identity]): Key
}
\end{minted} 
For example, we would represent \mintinline{scala}{KeyId} or \mintinline{scala}{Fingerprint} values as simple \mintinline{scala}{UUID} instances. This allowed us to work with values that were unlikely to collide with each other while keeping the implementation relatively simple:
\begin{minted}{scala}
    sealed trait KeyId {
        def value: String
    }
    case class KeyIdImpl(id: String) extends KeyId {
        def value: String = id
    }
    object KeyId {
        def random: KeyId = KeyIdImpl(UUID.randomUUID().toString)
    }
\end{minted}

When communicating with the actual Hagrid implementation, that solution is obviously insufficient because Hagrid verifies all incoming PGP keys for proper formatting and structure. To solve this problem, we decided to implement a mechanism capable of generating valid PGP keys. 
For this purpose, we use \emph{BouncyCastle}\footnote{https://www.bouncycastle.org/}, which is a Java library for cryptographic use cases.

We define a new method that expects a set of identities and returns a tuple consisting of the BouncyCastle encoding of a public key and a string containing the armored key text:
\begin{minted}{scala}
def genPublicKey(identities: Set[Identity]): (PGPPublicKey, String)
\end{minted}

Building upon this method, we can now define a new convenience method for constructing valid PGP keys given a set of identities: 
\begin{code}
    \begin{minted}{scala}
    def pgp(ids: Set[Identity]): Key = {
        val (pgpKey, armored) = KeyGenerator.genPublicKey(ids)
    
        PGPKey(
            keyId = KeyIdImpl(pgpKey.getKeyID.toString),
            fingerprint = FingerprintImpl(
                convertBytesToHex(
                    pgpKey.getFingerprint
                ).toUpperCase),
            identities = ids,
            armored = armored
        )
    }
    \end{minted}
\end{code}
It should be noted though that we currently do not support the generation of keys with arbitrary amounts of identities. This is due to several unsolved problems which we encountered during development. In the end, these hurdles led to the decision that temporarily supporting a maximum amount of only \emph{two} identities is acceptable. This restriction is currently implemented in the generator types mentioned in chapter \ref{sec:test_approach} which limit the amount of identities per key to a maximum value of two.
\paragraph{Running tests against a local Hagrid instance} Given that our test approach outlined in chapter \ref{sec:test_approach} is fully agnostic towards the underlying server implementation, adding support for Hagrid only required us to adhere to the API contract given by \mintinline{scala}{HagridInterface}.
\subsection{Initializing}
One major challenge that we faced when trying to incorporate the actual HAGRID implementation in our testing approach was the fact that HAGRID maintains internal state across several distinct instances. This is due to HAGRID persisting uploaded keys and their contents within its installation folder. 
In contrast to this, our high-level model of HAGRID maintains no state at all between any two separate test runs.

This is problematic, because our testing approach fundamentally relies on a limited amount of \emph{Identities}, that are subsequently used to construct PGP Keys. This would necessarily lead to unexpected results when applying our approach to HAGRID, where there could potentially already be an identity associated to some PGP key, that was uploaded during a previous testing session. 
To elaborate on this problem, let us assume these two separate test runs in pseudo-code:

Assuming that there exists a value \mintinline{scala}{key1 = Key(identity1,identity2)} and a value \mintinline{scala}{key2 = Key(identity3,identity4)}
 where all arguments to the \mintinline{scala}{Key} constructor are instances of \mintinline{scala}{Identity} with an arbitrary email address, we can define the following two test runs: 
\begin{figure}[!h]
    \begin{minipage}{0.5\textwidth}
        \centering
        \begin{minted}{scala}
        upload(key1)
        verify(identity1)
        \end{minted}
        \title{listing}{\textbf{Test run 1}}
    \end{minipage}
    \begin{minipage}{0.5\textwidth}
        \centering
        \begin{minted}{scala}
        verify(identity1)
        upload(key2)
        \end{minted}
        \title{listing}{\textbf{Test run 2}}
    \end{minipage}
\end{figure}

Looking \emph{only} at test run 2 we would expect an outcome, in which the server returns no public keys/identities at all. Instead, when we actually execute these two runs in sequential order, we receive a PGP key when querying the server with \mintinline{scala}{identity1}, even though we \emph{never} uploaded it during the second test. 
Any solution to this problem therefore requires us to reset the internal state of HAGRID after having completed a test.
This in turn also imposes an additional non trivial performance penalty on the execution speed of our ScalaCheck test, besides the slowdown caused by having to generate real PGP keys. 
For our current solution to this problem, we decided to compile HAGRID into a static image. This allows us to create a fresh instance of HAGRID at the beginning of each test iteration and shutting it down afterwards, at which point we simply clear the local directory of PGP keys. This ensures that each test run starts on a truly equal playing field.


