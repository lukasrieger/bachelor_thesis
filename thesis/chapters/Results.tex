\section{Results}

Our tests involving the abstract HAGRID model were able to show that our implementation appears to conform to our privacy specifications.
In all instances, our history evaluation would finish successfully: 
\begin{code}
    \begin{minted}{text}
+ History execution.historyWithSecureServer: OK, passed 10000 tests.
    \end{minted}
    \caption{Output of a successful test run with 10000 generated histories}
\end{code} 
\bigskip 
Conversely, if we replaced our correct model with an invalid one that does not filter unverified identities from keys, we were still able to consistently detect this incorrect behaviour. For example, this is the output generated by one of our test runs, utilizing an incorrect server model: 
\begin{code}
    \begin{minted}{text}
    [info] > ARG_1: History(
        Upload(
            PGPKey(KeyIdImpl(54ce7607-...),
            FingerprintImpl(65a7d3b4-...),
            Set(Identity(ilyaz@comcast.net), 
                Identity(ranasta@live.com))
        )), 
        Upload(
            PGPKey(KeyIdImpl(cf2a181c-...),
            FingerprintImpl(713bf494-...),
            Set(Identity(crowl@verizon.net), 
                Identity(wonderkid@yahoo.com))
        )), 
        Revoke(
            Set(),
            FingerprintImpl(713bf494-...)
        )
    )
    \end{minted}
    \caption{Generated history of a failing test run}
\end{code}

As we can see, our generated history consists of two \mintinline{text}{Upload} events and a single \mintinline{text}{Revoke} event. Given that the last \mintinline{text}{Revoke} event is effectively empty and no identities have been verified, we would expect all identities to be \mintinline{text}{Private}. Instead, our test evaluation shows us the following: 
\begin{code}
    \begin{minted}{text}
    Map (
        FingerprintImpl(713bf494-...) -> Map (
            Identity(crowl@verizon.net) -> 
                Mismatch(
                    Some((Identity(crowl@verizon.net),Private)),
                    Some(Identity(crowl@verizon.net))
                )
            Identity(wonderkid@yahoo.com) -> 
                Mismatch(
                    Some((Identity(wonderkid@yahoo.com),Private)),
                    Some(Identity(wonderkid@yahoo.com))
                )
        )
        FingerprintImpl(65a7d3b4-...) -> Map (
            Identity(ilyaz@comcast.net) -> 
                Mismatch(
                    Some((Identity(ilyaz@comcast.net),Private)),
                    Some(Identity(ilyaz@comcast.net))
                )
            Identity(ranasta@live.com) -> 
                Mismatch(
                    Some((Identity(ranasta@live.com),Private)),
                    Some(Identity(ranasta@live.com))
                )
        )
    )   
    \end{minted}
    \caption{Evaluation of a failed history run}
\end{code}

Indeed, our test has determined that our model of HAGRID has erroneously leaked \emph{all} identities of \emph{all} keys, even though they should have been \mintinline{text}{Private}. 

\paragraph{Results of our tests involving an actual HAGRID instance}
Perhaps unsurprisingly, we were not able to find any identity leaks when testing the real HAGRID implementation. Instead, every test we ran on HAGRID confirmed that the implementation does indeed conform to its claimed privacy constraints.
In fact, the only encounters with failing test results were caused by a number of subtle bugs in the way we communicated with HAGRID, such as the fact that HAGRID doesn't send duplicate verification mails for already verified identities.
