
\section{Practical implementation in Scala}
\label{sec:abstract_model}
\paragraph{The abstract server model}
In order to test the previously illustrated privacy constraints of PGP keys, we require a model of a PGP server which is sufficiently abstract to hide irrelevant implementation details, yet still exposes all functional specifications of the HAGRID server.

A minimal representation of a secure PGP server according the the HAGRID specifications would therefore require the following operations:
\begin{itemize}
    \item \emph{Request adding a key} - Request adding a new key to the database. This key may potentially contain an arbitrary amount of associated identities. The uploaded key is not made publicly available immediately after uploading. Instead, the server issues a confirmation code that may be used in subsequent confirmation requests.
    \item \emph{Confirm an addition} - Confirm a previously uploaded \mintinline{scala}{(Identity, Key)} pair given a valid confirmation code. This action does not directly confirm the selected identity but instead issues a confirmation email. The confirmation is only completed if the user uses the code contained within the email.
    \item \emph{get(ByMail, ByFingerprint, ByKeyId)} - Retrieve a key from the database given some identifying index (e.g.the fingerprint of the PGP key or one of the associated identities).
    \item \emph{Request a deletion} - Request the removal of some previously confirmed \mintinline{scala}{(Identity, Key)} pair. This action does not directly delete the selected identity but instead issues a confirmation email.
    \item \emph{Confirm a deletion} - Finalize the removal of some \mintinline{scala}{(Identity, Key)} pair. This operation requires the confirmation code, which the user must have obtained from a confirmation mail issued by the previous operation.
\end{itemize}
This set of requirements allows for a relatively direct translation into an abstract base trait: 
\begin{code}
    \begin{minted}{scala}
    trait HagridInterface {
        def upload(key: Key): Token

        def byEmail(identity: Identity): Option[Key]
        def byFingerprint(fingerprint: Fingerprint): Option[Key]
        def byKeyId(keyId: KeyId): Iterable[Key]

        
        def requestVerify(from: Token, 
                          emails: Set[Identity]): Seq[Body]
        def verify(token: Token)

        def requestManage(identity: Identity): Option[EMail]
        def revoke(token: Token, 
                   emails: Set[Identity])
    }
    \end{minted}
        \caption{Definition of the abstract server type}
        \label{def:HagridInterface}
\end{code}

The datatypes \mintinline{scala}{Fingerprint} and \mintinline{scala}{KeyId} represent their PGP counterparts as defined in the OpenPGP message format according to \cite{callas1998openpgp}. The \mintinline{scala}{Identity} type serves as a simple wrapper for a string containing an email address. \mintinline{scala}{Key} combines the previously mentioned types into what our server model will treat as a representation of a full PGP key.
\begin{code}
    \begin{minted}{scala}
    trait Key {
        def keyId: KeyId
        def fingerprint: Fingerprint
        def identities: Set[Identity]
    }
    \end{minted}
\caption{Definition of the abstract Key type}
\end{code}

Given that we only needed a high-level representation of HAGRID, our implementation does not persist internal state. Instead, the model state is simply kept as a number of maps whose elements transition between each other: 
\begin{code}
    \begin{minted}{scala}
    class Server extends HagridInterface {
        var keys: Map[Fingerprint, Key]
        var uploaded: Map[Token, Fingerprint]
        var pending: Map[Token, (Fingerprint, Identity)]
        var confirmed: Map[Identity, Fingerprint]
        var managed: Map[Token, Fingerprint]
        ...
    }
    \end{minted} 
    \caption{Representation of the internal server state}
\end{code}


\paragraph{Changing the internal server state}
Any of the actions exposed by the \mintinline{scala}{HagridInterface} (see definition in section \ref{def:HagridInterface}) may modify these internal state maps.
For example, we can inspect the implementation of the \mintinline{scala}{upload} method:
\begin{code}
    \begin{minted}{scala}
    def upload(key: Key): Token = {
        val fingerprint = key.fingerprint

        if (keys contains fingerprint)
        assert(keys(fingerprint) == key)

        val token = Token.unique
        keys += (fingerprint -> key)
        uploaded += (token -> fingerprint)
        token
    }
    \end{minted}
    \caption{Implementation of \mintinline{Text}{upload()}}
\end{code}

We can see that both the \mintinline{scala}{keys} map, as well as the \mintinline{scala}{uploaded} map, is being modified. Note, that once a key has been uploaded, it remains present in the corresponding map forever. Any additional modifications such as verifying or revoking an identity will then be tracked by entries in \mintinline{scala}{confirmed}.

\paragraph{Requesting verification of an identity} In contrast to simply uploading a key to the server, verifying or revoking an identity associated with some key is a two-step process.
For example, the process of verifying an identity consists of an initial call to \mintinline{scala}{requestVerify}:
\begin{code}
    \begin{minted}{scala}
    def requestVerify(from: Token, 
                      identities: Set[Identity]): Seq[Body] = {
        if (uploaded contains from) {
        val fingerprint = uploaded(from)
        val key = keys(fingerprint)
        if (identities subsetOf key.identities)
            identities
            .map(identity => {
                val token = Token.unique
                pending += (token -> (fingerprint, identity))
                val email = Body(fingerprint, token, identity)
                email
            })
            .toSeq
        else Nil
        } else Nil
  }
    \end{minted}
    \caption{Requesting verification of a set of identities}
\end{code}
The method first checks whether the given set of identities actually belongs to the key associated with the supplied token. Assuming that this is the case, we then proceed to generate confirmation emails for each separate identity. This sequence of emails is then returned to the caller.

In a second step, one may then proceed to call \mintinline{scala}{verify}, which implements the actual verification logic: 
\begin{code}
    \begin{minted}{scala}
        def verify(token: Token) {
            if (pending contains token) {
                val (fingerprint, identity) = pending(token)
                pending -= token
                confirmed += (identity -> fingerprint)
            }
  }
    \end{minted}
    
\end{code}
In this case, the method expects a valid token object, which can be obtained from the mail returned by \mintinline{scala}{requestVerify()}: 
\begin{code}
    \begin{minted}{scala}
    case class EMail(message: String, 
                     fingerprint: Fingerprint, 
                     token: Token
                    )
    \end{minted}
    \caption{Abstract representation of HAGRID emails}
\end{code}

\paragraph{Requesting revocation of a verified identity} The process of revoking some previously verified identities follows the same pattern. First, one must issue a call to \mintinline{scala}{requestManage()}: 
\begin{code}
    \begin{minted}{scala}
        def requestManage(identity: Identity): Option[EMail] = {
            if (confirmed contains identity) {
                val token = Token.unique
                val fingerprint = confirmed(identity)
                managed += (token -> fingerprint)
                val email = EMail("manage", fingerprint, token)
                return Some(email)
            }
            None
  }
    \end{minted}
\end{code}
One important difference to the verification process is that revocation happens on a per-key basis. That means that we can request a revocation token for a specific identity and then use that token to revoke several identities as long as they belong to the same key: 
\begin{code}
    \begin{minted}{scala}
        def revoke(token: Token, identities: Set[Identity]) {
            if (managed contains token) {
                val fingerprint = managed(token)
                val key = keys(fingerprint)
                if (identities subsetOf key.identities) {
                    confirmed --= identities
                }
            }
  }
    \end{minted}
\end{code}

\paragraph{Filtering unverified identities from keys}
Keeping track of which identities are verified at any given time alone is not sufficient. Instead, we also need to ensure that no unverified identity is ever returned by our model in response to a key request. 

Therefore, we introduce a method \mintinline{scala}{filtered} which is responsible for excluding any unverified identity from the given key: 
\begin{code}
    \begin{minted}{scala}
        private def filtered(key: Key): Key = {
            def confirmedByFingerprint(key: Key) =
            (for ((ident, fingerprint) <- 
                    confirmed if key.fingerprint == fingerprint)
                yield ident).toSet
        
        key restrictedTo confirmedByFingerprint(key)
          }
    \end{minted}
\end{code}
which subsequently calls \mintinline{scala}{restrictedTo}, returning a new \mintinline{scala}{PGPKey} instance containing only verified identities:
\begin{code}
    \begin{minted}{scala}
    def restrictedTo(ids: Set[Identity]): Key = {
        assert(ids subsetOf identities)
        PGPKey(keyId, fingerprint, ids, armored)
    }
    \end{minted}
\end{code}


\paragraph{Invariants} 
Internally, we defined several consistency invariants to ensure that our server model always behaves correctly. 
Any state manipulating action exposed by the server will trigger a re-evaluation of said invariants.

\begin{enumerate}
    \item A key is valid if its fingerprint is registered in the \mintinline{scala}{keys} map.
    \label{invariant:key}
    \begin{minted}{scala}
        for ((fingerprint, key) <- keys) {
        assert(key.fingerprint == fingerprint)
        }
    \end{minted}
    \item Upload tokens must refer to a valid fingerprint that itself refers to a valid key (see invariant \ref{invariant:key})
    \begin{minted}{scala}
        for ((token, fingerprint) <- uploaded) {
            assert(keys contains fingerprint)
        }
    \end{minted}
    \item Pending validations must always refer to a valid key. Additionally, all pending validations must be for identity addresses that refer to the respective key.
    \begin{minted}{scala}
        for ((token, (fingerprint, identity)) <- pending) {
            assert(keys contains fingerprint)
            val key = keys(fingerprint)
            assert(key.identities contains identity)
          }
    \end{minted}
    \item All confirmed identity addresses must refer to valid keys. Additionally, all confirmed identities must be valid for the associated key.
    \begin{minted}{scala}
        for ((identity, fingerprint) <- confirmed) {
          assert(keys contains fingerprint)
          val key = keys(fingerprint)
          assert(key.identities contains identity)
        }
    \end{minted}
    \item All issued management tokens must refer to valid keys.
    \begin{minted}{scala}
        for ((token, fingerprint) <- managed) {
            assert(keys contains fingerprint)
          }
        }
    \end{minted}
\end{enumerate}


\subsection{Simulating actions and responses with histories}
The server model itself cannot execute any concrete actions without some accompanying actor that causes this action. Generally, we wanted to express the following abstract actions in a composable way: 
\begin{itemize}
    \item \emph{Upload some key \textbf{k} to the server}
    \item \emph{Verify some identity \textbf{i} in relation to some parent key \textbf{k}}
    \item \emph{Revoke some verified identity \textbf{i} from its parent key \textbf{k}}
    \item \emph{Request some key \textbf{k} from the server, using one of its identifying characteristics}
\end{itemize}

\paragraph{Arbitrarily sized histories of events}
As outlined in chapter \ref{sec:history_def}, we chose to denote the sequence of actions that should be simulated on the abstract server model through a ``history'' of abstract events.
Through this approach, we can define a high-level model of \emph{what} operations should be executed in a specific order while keeping the \emph{how} - the concrete execution strategy - completely separate.

Once again, the different kinds of events are closely related to the capabilities exposed by our server model. Specifically, we recognize three distinct event types, that are modeled as subclasses of a \mintinline{scala}{sealed trait Event}: 
\begin{code}
    \begin{minted}{scala}
    case class Revoke(ids: Set[Identity], fingerprint: Fingerprint) 
        extends Event
    
    case class Upload(key: Key) extends Event
    
    case class Verify(ids: Set[Identity], fingerprint: Fingerprint) 
        extends Event
    \end{minted}
    \caption{Event type implementation as defined in chapter \ref{sec:history_def}}
\end{code}

A complete history then simply acts as a lightweight wrapper around a buffer of \mintinline{scala}{Event} objects:
\begin{minted}{scala}
case class History(events: mutable.Buffer[Event] = mutable.Buffer())
\end{minted}

\subsection{Executing a history on the server model}
Given a history and the computed privacy state based on that history, we can now define a simple execution strategy that sequentially loops through the history and sends the corresponding data to the server. The signature of the execution algorithm is given by \mintinline{scala}{def execute(server: HagridInterface, history: History): Unit}.
The algorithm then simply pattern matches the concrete event type and based on that type, calls the corresponding server method: 
\begin{code}
    \begin{minted}{scala}
    val uploaded: mutable.Map[Fingerprint, Token] = mutable.Map()
    for (event <- history.events) {
        event match {
        case Event.Upload(key) =>
            uploaded += ((key.fingerprint, server upload key))
        case Event.Revoke(identities, _) =>
            for {
                head <- identities
                token <- server requestManage head map (_.token)
            } server revoke(token, Set(head))
        case Event.Verify(identities, fingerprint) =>
            for {
                uploadToken <- uploaded.get(fingerprint)
                Body(_, token, _) <- server
                    .requestVerify(uploadToken, identities)
            } server verify token
        }
    }
    \end{minted}
    \caption{Sequential execution of a history}
\end{code}

As mentioned in chapter \ref{sec:history_def}, the history itself is completely unaware of \emph{how} it will be interpreted. In fact, the history is intentionally abstracting over any concrete execution details like having to follow the two-step verification process of first requesting a confirmation email and then completing the actual verification.


\subsection{History evaluation strategies}
Given some history \emph{h} we now want to determine precisely which associations between identities and keys should be visible depending on the sequence of events in \(h\).
To determine the visibility of an identity at any time, we introduce a \mintinline{scala}{State} type that tags the current state of an identity throughout the evaluation of the history.
The state of an identity may take any of these three forms: 
\begin{itemize}
    \item Public: The identity has been confirmed and is therefore publicly visible
    \item Private: The identity has not been confirmed yet and therefore remains private
    \item Revoked: The identity had been confirmed at some prior point but has since been revoked.
\end{itemize}
In terms of visibility, a \mintinline{scala}{Private} identity and a \mintinline{scala}{Revoked} identity can be considered equal from the perspective of a user. In both cases, the identity should not be included in any key requests.

In order to determine these states for every possible identity, we define a new method \mintinline{scala}{def states: Map[Fingerprint, Set[(Identity, Status)]]} within \mintinline{scala}{History}.
The method starts off with three separate maps \mintinline{scala}{Uploaded}, \mintinline{scala}{Confirmed} and \mintinline{scala}{Revoked}. These maps are then populated by folding over the sequence of events.
Depending on the currently folded value, one or several of the maps will be updated. 

\begin{code}
    \begin{minted}{scala}
val withUploadedAndConfirmed = confirmed.foldLeft(withUploaded) { 
    (acc, elem) =>
        acc.get(elem._2) match {
            case Some(states) =>
                acc updated(elem._2, 
                        states - (elem._1 -> Private) 
                               - (elem._1 -> Revoked) 
                               + (elem._1 -> Public)
                        )
            case _ => acc
        }
}.toMap
    \end{minted}
    \caption{Adding confirmed identities to their associated key}
\end{code}

We sequentially fold over these maps in order to ensure that \mintinline{scala}{Verify} or \mintinline{scala}{Revoke} instructions that refer to non-existant or non-public keys are ignored.


In a second step, this mapping from keys to \mintinline{text}{(Identity,Status)} tuples is compared to the state of all available identities coming from the server model.
To this end, we collect the total server state in a \mintinline{scala}{Map[Fingerprint,Set[Identity]]}, mapping each requested key to the set of identities we received. 
Next, we simply compare this set of identities to the corresponding set of \mintinline{text}{(Identity,Status)} tuples which we determined earlier. The result of this comparison is encoded in a \mintinline{scala}{EvalResult} type, which is a discriminated union between of the types \mintinline{scala}{Ok} and \mintinline{scala}{Mismatch}: 
\begin{code}
    \begin{minted}{scala}
    (state, serverState) match {
        case (None, None) => EvalResult.Ok
        case (None, Some(id)) => EvalResult.Mismatch(None, Some(id))
        case (Some((id, Private)), None) => EvalResult.Ok
        case (Some((id, Public)), None) =>
            EvalResult.Mismatch(Some((id, Public)), None)
        case (Some((id, Revoked)), None) => EvalResult.Ok

        case (Some((identity, Private)), Some(id)) =>
            EvalResult.Mismatch(Some((identity, Private)), Some(id))
        case (Some((identity, Public)), Some(id)) => EvalResult.Ok
        case (Some((identity, Revoked)), Some(id)) =>
            EvalResult.Mismatch(Some((identity, Revoked)), Some(id))
        }
    \end{minted}
    \caption{Comparison between a computed state and a server state}
\end{code}

In this case, the left side of our \mintinline{scala}{Mismatch} type contains the information we computed by evaluating the history, while the right side contains the server response. For example, the case in which we determined that a given Identity is \mintinline{text}{Private} but was still included in the server response would be encoded in the following way: \mintinline{scala}{Mismatch(Some((identity, Private)), Some(id))}

Finally, all these steps are combined in a single method \mintinline{scala}{check()} which returns a \mintinline{scala}{Map[Fingerprint, Map[Identity, EvalResult]]} and basically encodes the \(\text{Valid}\) and \(\text{Invalid}\) sets we introduced in chapter \ref{sec:history_def} using a different format. 


\subsection{Generating and verifying histories with ScalaCheck}
\label{sec:test_approach}
In order to ensure that our dynamic testing approach covers the largest possible space of possibly valid or invalid interactions with HAGRID we were unable to rely on some handwritten (and therefore non-exhaustive) set of predetermined test cases. Instead, our goal was to be able to generate diverse kinds of histories of arbitrary length and content, which would allow us to minimize the risk of missing some obscure edge-case in the behaviour of HAGRID.

\paragraph{Generating test data with ScalaCheck generators}
We achieve this by utilizing a property-based testing approach, in which all our test cases depend on a \emph{generator} of arbitrary histories. The basic functionality for generators is provided by ScalaCheck\footnote{https://www.scalacheck.org/}, which exposes an abstract type \mintinline{Scala}{Gen[T]} for some type \mintinline{Scala}{T}. This type can subsequently be used as a potentially infinite source of test data, over which we can then express some universally quantified property that this data should possess.
For example, the following code encodes the property that the act of taking the square root of any integer times itself should result in the original integer value: 
\begin{minted}{Scala}
    val propSqrt = forAll { (n: Int) => 
        scala.math.sqrt(n*n) == n 
    }
\end{minted}

We can now verify whether this property holds by calling \mintinline{Scala}{propSqrt.check}. 
ScalaCheck will automatically check, whether this property holds for some sufficiently large amount of integer values, which are provided by a built-in integer generator. The result of executing the property check will confirm to us, what we already know -- namely that this property is obviously not true for any negative integer value: 
\begin{code}
    \begin{minted}{Scala}
    scala> propSqrt.check
    ! Falsified after 2 passed tests.
    > ARG_0: -1
    > ARG_0_ORIGINAL: -488187735
    \end{minted} 
    \caption{ScalaCheck sample output}
\end{code}

What this output tells us, is that the first value found to falsify our property was \(n=-488187735\). ScalaCheck was then able to \emph{shrink} this failing instance down to the much simpler case of \(n=-1\).

\paragraph{Implementing a generator for histories} 
While ScalaCheck does provide a set of built-in generators for common datatypes such as \mintinline{scala}{Int}, generators for more complex, custom types must be implemented by hand. Our history generator consists of two components: A generator for ``contexts'' which hold the set of available PGP keys and finally a generator for event sequences. 
We generate such context objects, to ensure that all events within a single history refer to a consistent set of common PGP keys and identities:
\begin{minted}{Scala}
    class Context(val keys: Map[Fingerprint, Key])

    def contextGen(keySize: Int, idsPerKey: Int)
                  (keyGen: Int => Gen[Key]): Gen[Context] = 
        for {
            keys <- Gen.listOfN(keySize,keyGen(keySize))
            keyMap = (keys map (k => (k.fingerprint, k))).toMap
        } yield new Context(keyMap)
\end{minted}

The method expects an implicit parameter \mintinline{scala}{keyGen} which handles the generation of PGP keys. Adding this abstraction over the concrete key generator is beneficial because it allows us to produce histories independent from the underlying structure of the PGP keys. As we will see in chapter \ref{sec:real_hagrid}, this is particularly useful, when the requirements for PGP keys change based on whether we are testing our abstract model or the actual HAGRID implementation.

In the case of our abstract server model, the implementation of the key generator looks as follows: 
\begin{code}
    \begin{minted}{scala}
    def identityGen: Gen[Identity] =
        for (mail <- Gen.oneOf(Identity.mails)) 
            yield Identity(mail)
    
    def identitySetGen(size: Int): Gen[Set[Identity]] =
        for (identities <- Gen.listOfN(size, identityGen)) 
            yield identities.toSet

    def keyGen(idSize: Int): Gen[Key] =
        for (id <- identitySetGen(idSize)) yield Key.random(id)
    \end{minted}
\end{code}

We randomly choose a fixed number of identities from the predefined set \mintinline{scala}{Identity.mails} and pass them on to \mintinline{scala}{Key.random}: 
\begin{code}
    \begin{minted}{scala}
    def random(ids: Set[Identity]): Key =
        PGPKey(
            keyId = KeyId.random,
            fingerprint = Fingerprint.random,
            identities = ids,
            armored = ""
        )
    \end{minted}
    \caption{Method for generating a random PGP key}
\end{code}

Note that the key returned by this method is not a true PGP key, as both \mintinline{scala}{KeyId.random} as well as \mintinline{scala}{Fingerprint.random} simply use \mintinline{scala}{UUID.randomUUID()} under the hood. 

In a second step, our generator produces a random sequence of events, in which all necessary key data is provided by the current context.
\begin{code}
    \begin{minted}{Scala}
    def uploadEventGen(implicit context: Context): Gen[Event] =
        for ((_, key) <- Gen.oneOf(context.keys)) 
            yield Event.Upload(key)


    def verifyEventGen(implicit context: Context): Gen[Event] =
        for {
            (fingerprint, key) <- Gen.oneOf(context.keys)
            identities <- Gen.someOf(key.identities)
        } yield Event.Verify(identities.toSet, fingerprint)


    def revokeEventGen(implicit context: Context): Gen[Event] =
        for {
            (fingerprint, key) <- Gen.oneOf(context.keys)
            identities <- Gen.someOf(key.identities)
        } yield Event.Revoke(identities.toSet, fingerprint)
    \end{minted}
    \caption{Each event type has a separate generator}
\end{code}


Finally, we combine these separate event generators into a single generator that supports all three kinds of event types:
\begin{code}
    \begin{minted}{Scala}
    def eventGen(implicit context: Context): Gen[Event] =
        Gen.oneOf(uploadEventGen, verifyEventGen, revokeEventGen)
    \end{minted}
    \caption{Generator that randomly selects a specific event type}
\end{code}


The complete history generator then simply combines both context and event generators and wraps the generated sequences in a \mintinline{Scala}{History} object.
\begin{code}
    \begin{minted}{Scala}
    def historyGen(length: Int)
                  (implicit keyGen: Int => Gen[Key]): Gen[History] =
        for {
            context <- contextGen(2, 2)(keyGen)
            events <- Gen.listOfN(length, eventGen(context))
        } yield History(events.toBuffer)
    \end{minted}
    \caption{Complete history generator}
\end{code}

\paragraph{Verifying generated histories}
Given our implementation of a generator for histories, we can translate the formal condition for the validity of a HAGRID model from chapter \ref{sec:priv_condition}:
\[
    \forall h \in H \; .\; \text{Invalid}_h = \emptyset
\]
as a ScalaCheck property:
\begin{code}
    \begin{minted}{scala}
property("CheckHagridValid") = forAll { gen: Gen[History] =>
    forAll(gen) { history =>
        val server = new Server
        Sequential.execute(server, history)
        val (fingerprintResult, mailResult) = history.check(server)

        val successFingerprint = mailResult.forall {
            case (_, idMap) => idMap.forall(_._2 == EvalResult.Ok)
        }
        val successMail = fingerprintResult.forall {
            case (_, idMap) => idMap.forall(_._2 == EvalResult.Ok)
        }
        successFingerprint && successMail
    }
}
    \end{minted}
    \caption{Universally quantified validity constraint over histories}
    \label{code:propCheck}
\end{code}
The property seen in code sample \ref{code:propCheck} will only succeed if the evaluation of our history tags \emph{all} identities with \mintinline{scala}{EvalResult.Ok} values.

Note that instead of defining our property directly over an arbitrary history, we receive an instance of \mintinline{scala}{Gen[History]}. This allows us to define a ScalaCheck generator that in turn produces generators of histories of different lengths: 
\begin{code}
    \begin{minted}{scala}
    val MinHistorySize = 10
    val MaxHistorySize = 20

    def sizedHistoryGen(implicit keyGen: Int => Gen[Key]) =
        for {
            size <- Gen.choose(MinHistorySize, MaxHistorySize)
        } yield randomHistory(size)
    \end{minted}
    \caption{Nested generator for histories of different sizes}
\end{code}

This extra step of indirection thus allows us to check our privacy constraint over histories of different event sequences as well as different sizes.
\paragraph{Ensuring that invalid server behaviour is recognised}
In order to verify that this property is capable of detecting invalid behaviour on the server-side, we define a second property:
\begin{code}
    \begin{minted}{scala}
property("CheckHagridInvalid") = forAll { gen: Gen[History] =>
    forAll(gen) { history =>
        val server = new ServerOld
        Sequential.execute(server, history)
        val (fingerprintResult, mailResult) = history.check(server)

        val successFingerprint = mailResult.forall {
            case (_, idMap) => idMap.forall(_._2 == EvalResult.Ok)
        }
        val successMail = fingerprintResult.forall {
            case (_, idMap) => idMap.forall(_._2 == EvalResult.Ok)
        }
        successFingerprint && successMail
    }
}
    \end{minted}
    \caption{Property that tests a faulty server implementation}
\end{code}  

In this case, we create an instance of the type \mintinline{scala}{ServerOld}, which behaves the same way as the standard server implementation, except that it does not filter the identities of its keys. As seen later on in chapter \ref{sec:results}, this property reliably recognizes the release of private identities.

