
\section{Practical implementation in Scala}
\paragraph{The abstract server model}
In order to test the previously illustrated privacy constraints of PGP keys, we require a model of a PGP server which is sufficiently abstract to hide irrelevant implementation details, yet still exposes all functional specifications of the HAGRID server.

A minimal representation of a secure PGP server according the the HAGRID specifications would therefore require the following operations:
\begin{itemize}
    \item \emph{Request adding a key} - Request adding a new key to the database. This key may potentially contain an arbitrary amount of associated identities. The uploaded key is not made publicly available immediately after uploading. Instead, the server issues a confirmation code that may be used in subsequent confirmation requests.
    \item \emph{Confirm an addition} - Confirm a previously uploaded \mintinline{scala}{(Identity, Key)} pair given a valid confirmation code. This action does not directly confirm the selected identity but instead issues a confirmation email. The confirmation is only completed if the user uses the code contained within the email.
    \item \emph{get(ByMail, ByFingerprint, ByKeyId)} - Retrieve a key from the database given some identifying index (e.g.the fingerprint of the PGP key or one of the associated identities).
    \item \emph{Request a deletion} - Request the removal of some previously confirmed \mintinline{scala}{(Identity, Key)} pair. This action does not directly delete the selected identity but instead issues a confirmation email.
    \item \emph{Confirm a deletion} - Finalize the removal of some \mintinline{scala}{(Identity, Key)} pair. This operation requires the confirmation code, which the user must have obtained from a confirmation mail issued by the previous operation.
\end{itemize}
This set of requirements allows for a relatively direct translation into an abstract base trait: 
\begin{code}
    \begin{minted}{scala}
        trait HagridInterface {
            def byEmail(identity: Identity): Option[Key]
            def byFingerprint(fingerprint: Fingerprint): Option[Key]
            def byKeyId(keyId: KeyId): Iterable[Key]
            def upload(key: Key): Token
            def requestVerify(from: Token, emails: Set[Identity]): Seq[Body]
            def verify(token: Token)
            def requestManage(identity: Identity): Option[EMail]
            def revoke(token: Token, emails: Set[Identity])
        }
        \end{minted}
        \caption{Definition of the abstract server type}
        \label{def:HagridInterface}
\end{code}

The datatypes \mintinline{scala}{Fingerprint} and \mintinline{scala}{KeyId} represent their PGP counterparts as defined in the OpenPGP message format according to \cite{callas1998openpgp}. The \mintinline{scala}{Identity} type serves as a simple wrapper for a string containing an email address. \mintinline{scala}{Key} combines the previously mentioned types into what our server model will treat as a representation of a full PGP key.
\begin{code}
    \begin{minted}{scala}
        trait Key {
            def keyId: KeyId
            def fingerprint: Fingerprint
            def identities: Set[Identity]
        }
    \end{minted}
\caption{Definition of the abstract Key type}
\end{code}

Given that we only needed a high-level representation of Hagrid, our implementation does not persist internal state. Instead, the model state is simply kept as a number of maps whose elements transition between each other: 
\begin{code}
    \begin{minted}{scala}
        class Server extends HagridInterface {
            var keys: Map[Fingerprint, Key]
            var uploaded: Map[Token, Fingerprint]
            var pending: Map[Token, (Fingerprint, Identity)]
            var confirmed: Map[Identity, Fingerprint]
            var managed: Map[Token, Fingerprint]
            ...
        }
    \end{minted} 
    \caption{Representation of the internal server state}
\end{code}


\paragraph{Changing the internal server state}
Any of the actions exposed by the \mintinline{scala}{HagridInterface} (see definition in section \ref{def:HagridInterface}) may modify these internal state maps.
For example, we can inspect the implementation of the \mintinline{scala}{upload} method
\begin{code}
    \begin{minted}{scala}
        def upload(key: Key): Token = {
            val fingerprint = key.fingerprint
    
            if (keys contains fingerprint)
            assert(keys(fingerprint) == key)
    
            val token = Token.unique
            keys += (fingerprint -> key)
            uploaded += (token -> fingerprint)
            token
      }
    \end{minted}
    \caption{Implementation of \mintinline{Text}{upload()}}
\end{code}

we can see that both the \mintinline{scala}{keys} map, as well as the \mintinline{scala}{uploaded} map is being modified.

\paragraph{Invariants} 
Internally, we defined several consistency invariants to ensure that our server model always behaves correctly. 
Any state manipulating action exposed by the server will trigger a re-evaluation of said invariants.

\begin{enumerate}
    \item A key is valid if its fingerprint is registered in the \mintinline{scala}{keys} map.
    \label{invariant:key}
    \begin{minted}{scala}
        for ((fingerprint, key) <- keys) {
        assert(key.fingerprint == fingerprint)
        }
    \end{minted}
    \item Upload tokens must refer to a valid fingerprint that itself refers to a valid key (see invariant \ref{invariant:key})
    \begin{minted}{scala}
        for ((token, fingerprint) <- uploaded) {
            assert(keys contains fingerprint)
        }
    \end{minted}
    \item Pending validations must always refer to a valid key. Additionally, all pending validations must be for identity addresses that refer to the respective key.
    \begin{minted}{scala}
        for ((token, (fingerprint, identity)) <- pending) {
            assert(keys contains fingerprint)
            val key = keys(fingerprint)
            assert(key.identities contains identity)
          }
    \end{minted}
    \item All confirmed identity addresses must refer to valid keys. Additionally, all confirmed identities must be valid for the associated key.
    \begin{minted}{scala}
        for ((identity, fingerprint) <- confirmed) {
          assert(keys contains fingerprint)
          val key = keys(fingerprint)
          assert(key.identities contains identity)
        }
    \end{minted}
    \item All issued management tokens must refer to valid keys.
    \begin{minted}{scala}
        for ((token, fingerprint) <- managed) {
            assert(keys contains fingerprint)
          }
        }
    \end{minted}
\end{enumerate}


\newpage
\subsection{Simulating actions and responses with histories}
The server model itself cannot execute any concrete actions without some accompanying actor that causes this action. Generally, we wanted to express the following abstract actions in a composable way: 
\begin{itemize}
    \item \emph{Upload some key \textbf{k} to the server}
    \item \emph{Verify some identity \textbf{i} in relation to some parent key \textbf{k}}
    \item \emph{Revoke some verified identity \textbf{i} from its parent key \textbf{k}}
    \item \emph{Request some key \textbf{k} from the server, using one of its identifying characteristics}
\end{itemize}

\paragraph{Arbitrarily sized histories of events}
As outlined in chapter \ref{sec:history_def}, we chose to denote the sequence of actions that should be simulated on the abstract server model through a ``history'' of abstract events.
Through this approach, we can define a high-level model of \emph{what} operations should be executed in a specific order while keeping the \emph{how} - the concrete execution strategy - completely separate.

Once again, the different kinds of events are closely related to the capabilities exposed by our server model. Specifically, we recognise three distinct event types, that are modelled as subclasses of a \mintinline{scala}{sealed trait Event}: 
\begin{code}
    \begin{minted}{scala}
    case class Revoke(ids: Set[Identity], fingerprint: Fingerprint) 
        extends Event
    
    case class Upload(key: Key) extends Event
    
    case class Verify(ids: Set[Identity], fingerprint: Fingerprint) 
        extends Event
    \end{minted}
    \caption{Event type implementation as defined in chapter \ref{sec:history_def}}
\end{code}

A complete history then simply acts as a lightweight wrapper around a buffer of \mintinline{scala}{Event} objects:
\begin{minted}[]{scala}
case class History(events: mutable.Buffer[Event] = mutable.Buffer())
\end{minted}

\subsection{History evaluation strategies}
Given some history \emph{h} we want to determine precisely which associations between identities and keys should be visible depending on the combination of events in the history.
To determine the visibility of an identity at any time, we introduce a \mintinline{scala}{State} type, that tags the current state of an identity throughout the evaluation of the history.
The state of an identity may take any of these three forms: 
\begin{itemize}
    \item Public: The identity has been confirmed and therefore publicly visible
    \item Private: The identity has not been confirmed yet and therefore remains private
    \item Revoked: The identity had been confirmed at some prior point in the history but has since been revoked.
\end{itemize}
In terms of visibility, a \mintinline{scala}{Private} identity and a \mintinline{scala}{Revoked} identity can be considered equal from the perspective of a user. In both cases, the identity should not be included in any key requests.

In order to determine these states for every possible identity, we define a new method \mintinline{scala}{def states: Map[Fingerprint, Set[(Identity, Status)]]} within \mintinline{scala}{History}.
The method starts off with three separate maps \mintinline{scala}{Uploaded}, \mintinline{scala}{Confirmed} and \mintinline{scala}{Revoked}. These maps are then populated by folding over the sequence of events.
Depending on the currently folded value, one or several of the maps will be updated. 

\begin{code}
    \begin{minted}{scala}
    val withUploadedAndConfirmed = confirmed.foldLeft(withUploaded) { (acc, elem) =>
        acc.get(elem._2) match {
            case Some(states) =>
                acc updated(elem._2, 
                        states - (elem._1 -> Private) - (elem._1 -> Revoked) + (elem._1 -> Public)
                        )
            case _ => acc
        }
    }.toMap
    \end{minted}
    \caption{Adding confirmed identities to their associated key}
\end{code}

\paragraph{Executing a history on the server model}
Given a history and the computed privacy state based on that history, we can now define a simple execution strategy that sequentially loops through the history and sends the corresponding data to the server.
The signature of the execution algorithm is given by \mintinline{scala}{def execute(server: HagridInterface, history: History): Unit}.
The algorithm then simply pattern matches the concrete event type and based on that type, calls the corresponding server method: 
\begin{code}
    \begin{minted}{scala}
        val uploaded: mutable.Map[Fingerprint, Token] = mutable.Map()
        for (event <- history.events) {
          event match {
            case Event.Upload(key) =>
              uploaded += ((key.fingerprint, server upload key))
            case Event.Revoke(identities, _) =>
              for {
                head <- identities
                token <- server requestManage head map (_.token)
              } server revoke(token, Set(head))
            case Event.Verify(identities, fingerprint) =>
              for {
                uploadToken <- uploaded.get(fingerprint)
                Body(_, token, _) <- server requestVerify(uploadToken, identities)
              } server verify token
          }
        }
    \end{minted}
    \caption{Sequential execution of a history}
\end{code}

As already mentioned in \ref{sec:history_def}, the history itself is completely unaware of \emph{how} it will be interpreted. In fact, the history is specifically asbtracting over any concrete execution details like having to receive and respond to confirmation mails.

\todo{This explanation is currently pretty bad and unfinished.}
\todo{Maybe illustrate this with a diagram? Visualize the maps and how an event can change one or more of them. }

\newpage

\subsection{Generating arbitrary histories with ScalaCheck}
In order to ensure that our dynamic testing approach covers the largest possible space of possibly valid or invalid interactions with HAGRID we were unable to rely on some handwritten (and therefore non-exhaustive) set of predetermined test cases. Instead, our goal was to be able to generate diverse kinds of histories of arbitrary length and content, which would allow us to minimize the risk of missing some obscure edge-case in the behaviour of HAGRID.

\subsubsection{Generating test data with ScalaCheck generators}
We achieved this by utilizing a property based testing approach, in which all our test cases depend on a \emph{generator} of arbitrary histories. The basic functionality for generators is provided by ScalaCheck, which exposes an abstract type \mintinline{Scala}{Gen[T]} for some type \mintinline{Scala}{T}. This type can subsequently be used as a potentially infinite source of test data, over which we can then express some universally quantified property that this data should possess.
For example, the following code encodes the property that the act of taking the square root of any integer times itself should result in the original integer value: 
\begin{minted}{Scala}
    val propSqrt = forAll { (n: Int) => 
        scala.math.sqrt(n*n) == n 
    }
\end{minted}

We can now check that this property actually holds true by calling \mintinline{Scala}{propSqrt.check}. 
ScalaCheck will now automatically check, whether this property holds true for some sufficiently large amount of integer values, which are provided by a built in integer generator. The result of executing the property check will confirm to us, what we already know -- namely that this property is obviously not true for any negative integer value: 
\begin{minted}{Scala}
scala> propSqrt.check
! Falsified after 2 passed tests.
> ARG_0: -1
> ARG_0_ORIGINAL: -488187735
\end{minted} 
What this output tells us, is that the first value found to falsify our property was \(n=-488187735\). ScalaCheck was then able to \emph{shrink} this failing instance down to the much simpler case of \(n=-1\).

\subsubsection{Implementing a generator for histories}
In order to ensure that all events within a single history refer to a consistent set of common PGP keys and identities, we first generate a context object that carries all relevant information: 

\begin{minted}{Scala}
    class Context(val keys: Map[Fingerprint, Key])

    def contextGen(keySize: Int, idsPerKey: Int)
                  (implicit keyGen: Int => Gen[Key]): Gen[Context] = 
        for {
            keys <- Gen.listOfN(keySize,keyGen(keySize))
            keyMap = (keys map (k => (k.fingerprint, k))).toMap
        } yield new Context(keyMap)
\end{minted}

In a second step, our generator produces a random sequence of events, in which all necessary key data is provided by the current context.

\begin{minted}{Scala}
    def uploadEventGen(implicit context: Context): Gen[Event] =
        for ((_, key) <- Gen.oneOf(context.keys)) yield Event.Upload(key)


    def verifyEventGen(implicit context: Context): Gen[Event] =
        for {
            (fingerprint, key) <- Gen.oneOf(context.keys)
            identities <- Gen.someOf(key.identities)
        } yield Event.Verify(identities.toSet, fingerprint)


    def revokeEventGen(implicit context: Context): Gen[Event] =
        for {
            (fingerprint, key) <- Gen.oneOf(context.keys)
            identities <- Gen.someOf(key.identities)
        } yield Event.Revoke(identities.toSet, fingerprint)
\end{minted}

Finally, we combine these separate event generators into a single generator that supports all three kinds of event types

\begin{minted}{Scala}
    def eventGen(implicit context: Context): Gen[Event] =
        Gen.oneOf(uploadEventGen, verifyEventGen, revokeEventGen)
\end{minted}

The complete history generator then simply combines both context and event generators and wraps the generated sequences in a \mintinline{Scala}{History} object.

\begin{minted}{Scala}
    def historyGen(length: Int)
                  (implicit keyGen: Int => Gen[Key]): Gen[History] =
        for {
            context <- contextGen(2, 2)(keyGen)
            events <- Gen.listOfN(length, eventGen(context))
        } yield History(events.toBuffer)
\end{minted}

\textbf{TODO: Explain role of keyGen: Int => Gen[Key]}

\subsection{Initial approach to simulating actions through ``actors''}
Our initial approach to simulating user actions was based on fine grained \emph{actors} that each encapsulated a specific capability. 
Using this approach, any of the aforementioned operations would be represented by a single actor that runs on a simple state machine.
The actor trait is defined as following:
\begin{minted}{scala}
    trait Actor {
        def canAct: Boolean
        def act(): Unit
      
        val inbox = mutable.Queue[Data]()
        def canReceive = inbox.nonEmpty
      
        def isActive = canAct || canReceive
      
        def handle(from: Actor, msg: Message)
        def handle(from: Actor, msg: Body)
        def send(to: Actor, msg: Message): Unit = 
            Network.send(this, to, msg)
        def send(to: Identity, msg: Body): Unit = 
            Network.send(this, to, msg)
      
        def register(identity: Identity): Unit = 
            Network.register(identity, this)
    }
\end{minted}
An actor could define a custom \mintinline{scala}{act()} method that would initialize the specific action of this actor. Any future interaction with another actor would then be handled by the actors \mintinline{scala}{handle()} method. 

\todo{TODO}: Describe why there were two different handle methods. What is Network? What was bad about this approach? Show execution strategy and why order of execution was a problem.