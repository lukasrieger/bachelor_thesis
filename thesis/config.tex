

\newcommand{\faculty}{Software Systems Computational Lab}
\newcommand{\thesisauthor}{Lukas Johannes Rieger}
\newcommand{\supervisor}{Prof. Dr. Gidon Ernst}
\newcommand{\thesistitle}{Information Flow testing of a PGP server}
\newcommand{\thesisabstract} {Lorem Ipsum}
\newcommand{\thesisauthorship} {I hereby confirm that I have written the accompanying thesis by myself, without contributions from any sources other than those cited in the text and acknowledgements. This applies also to all graphics, drawings, maps and images included in the thesis.}

\selectlanguage{english}



% 2 Aufteilen.
% Jig is kacke weil wegen dynamischen Policies
% Scala Model as Top Level Abschnitt.

% Information Flow Theory: Kleines Beispiel.
% Z.B. Mock Up Server mit einem einzigen Geheimnis. Kann man declassifien oder nicht.
% Dieses Beispiel beschreiben. Mit History ? 

% 3.3 wieder Top Level

% Server: Invarianten erklären!

% Actor approach kann mann rauslassen.

% Top Level Ansatz: Declassification UND RECLASSFICIATION UND ZWAR DYNAMISCH

% Revocation geht im paper von 2018

% Hehbnjdbnjfhg, Was darf ein Angreifer sehen?.

% Real HAGRID: Technische Schwiereigkeiten erwähnen!

% HAGRID statisch bauen und dann immer hoch- und runterfahren.

% Failing Durchlauf erklären!!

